\documentclass[12pt]{article}

\usepackage{amsmath}
\usepackage{amssymb}
\usepackage{calc}
\usepackage{units}
\usepackage{graphicx}
\usepackage[pdftex]{hyperref}
\usepackage{subfig}
\usepackage[margin=1in]{geometry}
\usepackage{listings}
\usepackage[numbers,sort&compress]{natbib}
\usepackage{bm}
\usepackage{paralist}
\usepackage[draft]{fixme}
\usepackage{textcomp}

\hypersetup{
  breaklinks=true,
  pdftitle={Kirchoff's Rules},
  pdfauthor={Kevin R. Lynch based on a lab by D.C.Jain}, 
  pdfsubject={Phyiscs, Electricity and magnetism},
  pdfkeywords={Kirchoff's Rules},
  pdflang={en-US},
}

\title{Kirchoff's Rules}
\author{}
%Kevin R. Lynch, based on an earlier lab by D.C.Jain
%\date{2012-02-16}
\date{}

\begin{document}

\maketitle

\section{Objectives}
\label{sec:objectives}

\begin{enumerate}
\item To verify Kirchoff's Loop rule for voltages, and
\item To verify Kirchoff's Junction rule for currents.
\end{enumerate}

%\newpage

\section*{Pre-Lab Exercises}

Answer these questions as instructed on Blackboard; make sure to
submit them before your lab session!

\begin{enumerate}
\item Kirchoff's Rules are defined in terms of loops and junctions.
  How would you define a ``loop'' in a circuit?  What is a junction?
\item What happens to the reading of the ammeter if you reverse the
  connections?
\item Recall that current is just the motion of electric charges.  The
  Junction rule is a practical manifestation of what more fundamental
  conservation law?  What conservation law implies the loop rule?
\item Write the junction equation for point $B$ of Figure~2.1.
\item Write the loop equation for the loop $AFEBA$ of Figure~2.1.
\end{enumerate}

\newpage

\section*{Post-Lab Exercises}

\begin{enumerate}
\item For the one-loop circuit of Unit~1, verify the loop and junction
  rules.  That is, using your data, check if the currents into and out
  of point $B$ in Figure~2.2 are conserved.  Further, determine
  whether the loop rule is satisfied in the loops $ABCDA$,
  $AB_1C_1DA$, and $BB_1C_1CB$.  To what level of precision are the
  rules satisfied (what's the uncertainty)?
\item For the two-loop circuit of Unit~2, verify the loop and junction
  rules. There are two junctions of interest here, $A$ and $B$, and a
  large number of loops; look at loops $AFEBA$, $ABCDA$, $AFEBCDA$,
  and $AFEBB_2C_2DA$.
  To what level of precision are the rules satisfied?
\item Discuss briefly whether you have met the objectives of the lab
  exercises. 
\end{enumerate}

\end{document}

%%% Local Variables: 
%%% mode: latex
%%% TeX-master: t
%%% End: 
