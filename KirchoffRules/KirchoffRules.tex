\documentclass[12pt]{article}

\usepackage{amsmath}
\usepackage{amssymb}
\usepackage{calc}
\usepackage{units}
\usepackage{graphicx}
\usepackage[pdftex]{hyperref}
\usepackage{subfig}
\usepackage[margin=1in]{geometry}
\usepackage{listings}
\usepackage[numbers,sort&compress]{natbib}
\usepackage{bm}
\usepackage{paralist}
\usepackage[draft]{fixme}
\usepackage{textcomp}
\usepackage{framed}

\hypersetup{
  breaklinks=true,
  pdftitle={Kirchoff's Rules},
  pdfauthor={Kevin R. Lynch based on a lab by D.C.Jain}, 
  pdfsubject={Phyiscs, Electricity and magnetism},
  pdfkeywords={Kirchoff's Rules},
  pdflang={en-US},
}

\title{Kirchoff's Rules}
\author{}
%Kevin R. Lynch, based on an earlier lab by D.C.Jain
%\date{2012-02-16}
\date{}

\begin{document}

\maketitle

\section{Objectives}
\label{sec:objectives}

\begin{enumerate}
\item To verify Kirchoff's Loop rule for voltages, and
\item To verify Kirchoff's Junction rule for currents.
\end{enumerate}

\section{Introduction}
\label{sec:introduction}

Gustav Kirchoff was a prominent 19\textsuperscript{th} century
Prussian physicist.  He is still well known for his work in both
circuit theory and spectroscopy.  He won every prominent award for
science in the middle part of the century, and is remembered along
with his spectroscopist colleague Robert Bunsen\footnote{Yes,
  \textit{that} Bunsen.} by the Bunsen-Kirchoff Award for
Spectroscopy.

The study of electricity was still in its infancy in the early half of
the century; you may recall that Georg Ohm only discovered the
relationship between voltage, current, and resistance in 1827.
Kirchoff was certainly familiar with this work, and he first
discovered what are now called Kirchoff's Circuit Laws during his
electrical studies while a student in the 1840s.  This later formed
the basis for his doctoral dissertation.

\section{Theory}
\label{sec:theory}

Kirchoff's Rules easy to state, but first we need to define a few
terms.  A \textit{loop} in a circuit is any closed path that travels
through one or more of the circuit elements and returns to its origin.
A \textit{junction} is any point where two or more components are
connected by a conducting path.  With these definitions, the Rules
themselves are simple to state:
\begin{enumerate}
\item Loop Rule: The directed sum of the voltage drops around any loop
  in a circuit must be zero:
  \begin{gather*}
    \sum V_i = 0\ .
  \end{gather*}
\item Junction Rule: The directed sum of all currents entering a
  junction in any circuit must be zero:
  \begin{gather*}
    \sum I_i = 0\ .
  \end{gather*}
\end{enumerate}
The \textit{voltage drop} is the difference in voltage from one
terminal of a circuit element to another terminal.  By using the term
\textit{directed sum}, we indicate that the \textit{sign} of the term
matters: if we choose to define a current entering a junction as
positive, then one that leaves a junction must be negative; if we
choose to define the current as traveling in a clockwise direction
around a loop, then we should measure the voltage drops in the same
direction.

Kirchoff's Rules are equivalent to the conservation of energy and the
conservation of charge respectively.  The junction rule is easy to
understand in this context.  Since current is just the motion of
charged particles, the Junction Rule says simply that every charge
that goes into a junction must leave it: you can't create or destroy
charges.  The Loop Rule is a little more complicated, but not much:
because generating a voltage in a power supply requires a power
source, and driving a current through a resistor dissipates power, all
the energy that we add to the circuit must be consumed by the
resistive elements.

Kirchoff's Rules apply to more than just resistors and power supplies
(as we will learn in later labs), but they are really just
approximations to the full theory of electromagnetism described by
Maxwell's Equations.  Luckily for you, these exceptions do not arise
in any experiments we will perform in this class.

\begin{figure}
  \centering
  \includegraphics[width=\textwidth/2]{figures/junction}
  \caption{A junction, $A$, with four currents: $I_1$, $I_2$, $I_3$,
    and $I_4$.}
  \label{fig:junction}
\end{figure}
The \textit{use} of these rules is relatively straightforward, but
let's do some simple examples to make those uses concrete.  Let's
start with the Junction Rule.  When we measure the currents
\textit{into} a junction, we mean that we \textit{assume} that
currents along all branches flow into the junction, and make our
measurements assuming that's true.  But since that can't be true -
That's what the rule says, after all! - sometimes we'll measure the
current along a branch as \textit{negative}, which means that it
actually flows \textit{out} of the junction.  Take
Figure~\ref{fig:junction} as an example: there are four branches
connected to junction $J$.  Let's label the currents from points
$T_1$, $T_2$, etc by $I_1$, $I_2$, etc, and let's further assume that
they all flow \textit{into} $J$.  When we measure the current with our
ammeter, we have to choose which terminal of the ammeter to place
where.  By the conventional choice of inflowing currents, we should
\textit{always} connect the negative terminal of the ammeter at $J$,
while connecting the positive terminal to the $T_i$.  Then, the meter
will sometimes read positive values (for properly inflowing currents),
and sometimes read negative values (for outflowing currents), and the
sum of the total should be zero.  There are other possible approaches
to the measurement, but all of them must be equivalent to what we have
outlined here.

\begin{figure}
  \centering
  \includegraphics[width=\textwidth/2]{figures/loop}
  \caption{here be a caption}
  \label{fig:loop}
\end{figure}
The approach for Loop Rule measurements is almost as simple:  assume
that there is a current flowing around the loop in one direction or
the other, and measure all the voltage drops around the loop in that
direction.  So, for the loop in Figure~\ref{fig:loop}, starting with
node $A$, we'll assume a current flowing in a clockwise direciton.
First, measure the voltage drop between node $A$ and node $B$, by
measuring with the negative probe of the voltmeter at $A$, and the
positive probe at $B$.  Then, measure the voltage drop between nodes
$B$ and $C$ by placing the negative probe at $B$, and the positive
probe at $C$.  Repeat all the way around the loop.  Some of the
voltage drops will be negative and some positive, while the
accumulated voltage drop should vanish.  As for the Loop Rule, there
are other possible equivalent approaches to making this measurement,
but they must all be equivalent to what we have outlined here.

\section{Procedures}
\label{sec:procedures}

We will now investigate Kirchoff's Laws quantitatively, in two steps:
first, we will use a single loop circuit to study the loop rule in
isolation, and then we will add a second loop to study the loop and
junction rules at the same time.  For the experiment, you will need
three multimeters, two power supplies, the circuit board, and various
lengths of cabling.  Use the two identical multimeters exclusively as
ammeters, and the third multimeter for everything else.

\begin{figure}
  \centering
  \includegraphics[width=\textwidth/2]{figures/circuit_board}  
  \caption{The circuit board used in this lab.  You only need to use
    three of the resistors on this board; while you can use any three
    just as easily, we strongly suggest using the three parallel
    resistors at the bottom right.}
  \label{fig:circuitboard}
\end{figure}
\begin{framed}
  \begin{center}
    {\Large \textbf{Reminder}}
  \end{center}
  When we say ``Record a measurement'', we \textit{always} mean that
  you should record the measurement, \textit{and an estimate of the
    uncertainty of that measurement}!  \textbf{Always!}
\end{framed}
Remember also that the \textit{direction} of your measurement matters:
if you swap the measurement leads, you change the \textit{sign} of the
measured current or voltage (why?).  The same holds for power
supplies.  You must come up with a measurement convention, and
consistently apply it, or you will get the wrong results.  The circuit
board has five resistors on it, but you will only use three of them
for this lab; you can choose any three, but it is probably easiest to
use the three parallel resistors in the bottom right corner of the
board; see Figure~\ref{fig:circuitboard}.

\subsection{One loop}
\label{sec:oneloop}

\begin{figure}
  \centering
  \includegraphics[width=\textwidth/2]{figures/oneloop}
  \caption{The circuit used in Section~\ref{sec:oneloop} to study
    Kirchoff's Loop Rule.} 
  \label{fig:oneloop}
\end{figure}

You will use the circuit in Figure~\ref{fig:oneloop} to study the Loop
rule; note that you will not use all the resistors on the board.  Do
\textit{not} build this circuit yet!.
\begin{enumerate}
\item Before connecting any components, make sure to turn the power
  supply voltage \textit{on both power supplies} down to zero!
\item Measure and record the resistance of each resistor on the
  board.  Do these measurements match the color code on each resistor?
  They may not, but if they are in the ball park, you may use them
  anyway. 
\item Calculate the maximum safe current for the circuit; determine
  which ammeter inputs you need to use so as not to blow the fuses in
  the ammeter!  When in doubt, start with the largest range available! 
\item Connect the components as shown in the schematic, and slowly
  turn up the voltage to a value that will not exceed the maximum safe
  current in any of the resistors.  Something like \unit[15]{V} should
  be a reasonable starting value.  \textbf{Important!}: If the ammeter
  reading ``jumps'' from zero to a very high (or offscale) value when
  you start turning up the voltage, disconnect the circuit and get
  help from your Instructor \ldots do this \textit{before} you smell
  smoke or blow the fuses.  Also, if the power supply ``refuses'' to
  produce a voltage high enough for your liking, get help from your
  instructor to adjust the current limiter in the power supply.
\item Measure and record the current in the loop with the ammeter.
  Remember that currents are \textit{directed}; this means you need to
  measure with the probe leads oriented consistently.
\item Measure and record the voltage drops around the loop: pick a
  junction as you ``reference point'', and march around the loop
  measuring the voltage drops across the power supply, the resistors
  and the ammeter.  Remember that the sum you will do is
  \textit{directed}; you should use the same orientation convention
  you used to measure the current.  You should find that the voltage
  drop across the power supply has opposite sign to the voltage drops
  across all the other components.
\item Repeat the measurements at another voltage to check for
  consistency.
\end{enumerate}
You should at this point quickly check your data for consistency with
Kirchoff's Loop Rule.

Before moving on, reduce the power supply voltage back to zero.

\subsection{Two loops}
\label{sec:twoloops}

\begin{figure}
  \centering
  \includegraphics[width=\textwidth/2]{figures/twoloops}
  \caption{The circuit used in Section~\ref{sec:twoloops} to study
    Kirchoff's Loop and Junction Rules.} 
  \label{fig:twoloops}
\end{figure}

For this section, you will use the circuit in
Figure~\ref{fig:twoloops}; note that you will \textit{still} not use
all of the resistors on the board.
\begin{enumerate}
\item Deja Vu: Before connecting any components, make sure to turn the
  power supply voltage \textit{on both power supplies} down to zero!
\item Make sure that you in fact measured and recorded the resistances
  of each and every resistor on the board.
\item Connect the components as show in the schematic.  Be sure to
  match the polarities of both the power supplies and the ammeters.
\item Now, slowly turn up both power supplies to something like
  \unit[15]{V}.  Again, watch for indications that you have made a
  wiring error, and turn things off \textit{before} you break things,
  and \textit{before} getting help from your instructor.
\item Measure and record the currents in both ammeters.  You'll also
  need the current in the shared branch (through resistor $R_4$);
  you'll get the data for this in the next step.
\item Measure and record the voltage drops across both power supplies,
  all three resistors, and both ammeters.
\item Repeat the measurements two more times, with a different value
  of the voltage on one or the other of the power supplies.
\end{enumerate}
At this point, you should quickly check your data for consistency with
Kirchoff's Loop and Junction Rules.  You get the current through
resistor $R_4$ from Ohm's Law; watch the signs!

\newpage
\section*{Pre-Lab Exercises}

Answer these questions as instructed on Blackboard; make sure to
submit them before your lab session!

\begin{enumerate}
\item Kirchoff's Rules are defined in terms of loops and junctions.
  How would you define a ``loop'' in a circuit?  What is a junction?
\item What happens to the reading of the ammeter if you reverse the
  connections?  The voltmeter?
\item Recall that current is just the motion of electric charges.  The
  Junction rule is a practical manifestation of what more fundamental
  conservation law?  What conservation law implies the loop rule?
\item Write a junction equation for the point $B_1$ of
  Figure~\ref{fig:twoloops}.  Make clear your conventions for currents
\item Write a loop equation for the loop $BB_1C_1CB$ of
  Figure~\ref{fig:twoloops}. 
\end{enumerate}

\newpage

\section*{Post-Lab Exercises}

\begin{enumerate}
\item For the one-loop circuit of Section~\ref{sec:oneloop}, verify
  the Loop Rule.  Is the rule satisfied?  To what level of uncertainty?
\item For the two-loop circuit of Section~\ref{sec:twoloops}, verify
  both the Loop and Junction Rules.  There are two coupled loops, so
  you have three total loops to analyze: $BB_1C_1CB$,
  $B_1B_2C_2C_1B_1$, and $BB_1B_2C_2C_1CB$.  Are the rules satisfied
  in all cases?  To what level of uncertainty?
\item Discuss briefly whether you have met the objectives of the lab
  exercises. 
\end{enumerate}

\end{document}

%%% Local Variables: 
%%% mode: latex
%%% TeX-master: t
%%% End: 
