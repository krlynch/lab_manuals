\documentclass[12pt]{article}

\usepackage{graphicx}
\usepackage[margin=1in]{geometry}
\usepackage{units}
\usepackage{calc}
\usepackage{amsmath}
\usepackage{amssymb}
\usepackage{comment}
\usepackage{bm}
\usepackage{hyperref}
\usepackage{paralist}

\title{Some Ideas for PHYS152 Labs}
\author{Kevin Lynch}
\date{2012-01-13}

\begin{document}

\maketitle

\begin{abstract}
  We've been talking about replacing labs forever \ldots let's just do
  it.
\end{abstract}

\section{What we need}
\label{sec:need}

The course covers the following major topics that are amenable to
undergraduate teaching labs: electrostatics, simple DC and AC
electronics, magnetostatics, electromagnetics, ray and wave optics,
and (nominally) an introduction to modern physics.\footnote{We should
  officially abandon this last topic: we can't do this justice with
  the time constraints, and we need the hours elsewhere.  I wouldn't
  mind seeing it replaced with a brief introduction to Special
  Relativity, which is intimately connected with everything else in
  the course, and exciting for the students too!}  The laboratory
portion of the course has fourteen meetings, or which we ``devote''
the final session to a laboratory final.\footnote{Which, I remind you,
  I wouldn't be sad to see disappear.}

Ideally, we would have labs covering all of the major topics, along
with perhaps a dedicated lab to teach use of the major equipment
(multimeters, oscilloscopes, lasers, etc.).

We currently have a number of labs that are well past their prime, or
which are really not appropriate for the course: the Galvanometer, the
CRT, the Triode, and Digital Electronics.  We also have at least one
lab which requires a bit too much time, AC Circuits I (study of RC and
RL circuits).  

We're also missing labs in some of the major areas: electrostatics,
magnetostatics, electromagnetics, non-optical electromagnetic waves,
special relativity, etc.  In some areas, we are missing or not
properly emphasizing key concepts.  For instance, in ray optics, we
do lenses, but we don't explicitly do anything with reflection or
refraction.  In wave optics, we discuss transmission diffraction, but
not reflective diffraction.

Even for the labs we are keeping around, the lab manual text needs a
complete rewrite for our post-analog, post-CRT, digital, flat screen
era.

It would be nice to have some new equipment\footnote{That is,
  equipment designed and built after our students were born\ldots} in
particular, non-consumables that we can use in all the labs.
\begin{inparaitem}
  \item lots of cheap lasers,
  \item lots of good meters,
  \item modern oscilloscopes,
\end{inparaitem}

\section{What can we do quickly?}
\label{sec:quickly}

Quickly entails two things: we can get it done in time for this
semester, and we can afford to buy any ``inputs'' on the (virtually
non-existent) department budget.

We can certainly write new pre- and post-lab questions; I suggest we
rewrite these at least a week \textit{before} the lab first meets.

Let's take a look at the lab schedule for the last few semesters:
\begin{center}
\begin{tabular}{|c|l|}
\hline
\textbf{Lab \#} & \textbf{Experiment} \\ \hline
1 & Galvanometer \\ \hline
2 & Voltmeter and Ammeter \\ \hline
3 & Resistivity and Temperature Coefficient of Tungsten \\ \hline
4 & D.C. Circuits (Ohm's Law) \\ \hline
5 & Kirchhoff's Laws \\ \hline
6 & Cathode Ray Oscilloscope \\ \hline
7 & Characteristics of a Triode \\ \hline
8 & Time Constant of an RC Circuit \\ \hline
9 & A. C. Circuits I (RC and RL) \\ \hline
10 & A. C. Circuits II (RLC) \\ \hline
11 & Lenses \\ \hline
12 & Diffraction Grating \\ \hline
13 & Introduction to Digital Circuits and Logic Gates \\ \hline
14 & Lab Exam \\ \hline
\end{tabular}
\end{center}

I would propose dropping the Galvanometer, Triode, and Digital
Circuits labs, splitting AC I into ``The AC RC Circuit'' and ``The AC
RL Circuit''.  We still need two labs, so I suggest we do some more
ray and/or wave optics: reflection (``angle of incidence equals angle
of reflection''), reflective diffraction (bounce lasers off CD, DVD,
and BluRay disks to measure the ``feature size''), and refraction
(use lasers to directly measure the refractive index of mater,
acrylic, etc).  We could consider doing both the reflection tasks in
one lab.

I would also like to update the AC labs to use our BK Precision
multimeters to have the students actually measure the drive
frequencies, rather than guessing what they are off the function
generators and scopes.  This will require some minor rewrites of the
procedures for those labs \ldots but we're going to have to do that
any way to split the RC and RL labs.

This can all be done at very modest cost: we require a small number of
laser pointers,\footnote{From a safety perspective, we should either
  use the standard red laser pointers, or try to find some Class II
  green pointers.  I have a Class IIIa, and the thing actually scares
  me.  The green ones will probably be hard to find and expensive; not
  a short term item.} perhaps some acrylic\footnote{Or help from the
  geologists to polish the little chunks of acrylic rod that Joel has
  in his office.} or glass blocks or cubes, and ideally a small water
tank.  I have detailed thoughts in mind, but will have to sketch them
out in separate documents.

\section{What can we do in the medium term?}
\label{sec:midterm}

I would like us to completely rewrite the text of all the labs, and
present them in PDF form on the website; no printing ever again.  

We also need to find and implement decent magnetostatics and
electromagnetics labs, and perhaps a non-optical waves lab (radio,
microwave, etc).  I don't know what these should be, but whatever they
are, we will probably need more money than we can get out of the
department budget; that will require a trip to the Provost's office.
We need a decent plan.

\section{How about the long term?}
\label{sec:longterm}

In the long term, we need a major overhaul of our teaching lab
equipment.
\begin{itemize}
\item The oscilloscopes in particular are total dinosaurs that are
  finicky, and very hard to use\footnote{Even for an expert like me!};
  I really hate them.  We can replace them for roughly \$2000 - \$2500
  per unit.\footnote{Priced in Fall 2011 through LeCroy.}
\item We also need more, and more capable, multimeters; we have ten BK
  Precision units that do what I want, but we need at least twice that
  number; they're probably \$500 - \$1000 a piece.
\item Finally, every seat should have a computer.  It would be nice to
  have some computer based data acquisition, but at the very least, we
  should have spreadsheet based manually data recording.
\end{itemize}
Obviously, none of this is cheap.  We'll need to talk to the Provost,
and start searching for grant opportunities to fund it.  I just
remembered the RESO-A ``Legislative Funding Request'' process; we
dashed off something a few years ago that disappeared into a black
hole.  There may also be opportunities through the City and State
Departments of Education, and maybe private foundations.  Do we have
any rich Alums?

\end{document}


%%% Local Variables: 
%%% mode: latex
%%% TeX-master: t
%%% End: 
