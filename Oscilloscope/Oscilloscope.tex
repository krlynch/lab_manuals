\documentclass[12pt]{article}

\usepackage{amsmath}
\usepackage{amssymb}
\usepackage{calc}
\usepackage{units}
\usepackage{graphicx}
\usepackage[pdftex]{hyperref}
\usepackage{subfig}
\usepackage[margin=1in]{geometry}
\usepackage{listings}
\usepackage[numbers,sort&compress]{natbib}
\usepackage{bm}
\usepackage{paralist}
\usepackage[draft]{fixme}
\usepackage{textcomp}

\hypersetup{
  breaklinks=true,
  pdftitle={The Cathode Ray Oscilloscpe},
  pdfauthor={Kevin R. Lynch based on a lab by D.C.Jain}, 
  pdfsubject={Phyiscs, Electricity and magnetism},
  pdfkeywords={CRT, CRO, Cathode Ray, Oscilloscope},
  pdflang={en-US},
}

\title{The Cathode Ray Oscilloscope}
\author{}
%Kevin R. Lynch, based on an earlier lab by D.C.Jain
%\date{2012-02-23}
\date{}

\begin{document}

\maketitle

\section{Objectives}
\label{sec:objectives}

\begin{enumerate}
\item To study the operation of an oscilloscope, and
\item To measure DC and AC voltages, and the amplitude and frequency
  of a waveform with an oscilloscope.
\end{enumerate}

%\newpage

\section*{Pre-Lab Exercises}

Answer these questions as instructed on Blackboard; make sure to
submit them before your lab session!

\begin{enumerate}
\item The resistance between the input terminals of an oscilloscope
  are very large.  Does that make it more like a voltmeter or more
  like an ammeter?  Why?  Do you make measurements in parallel or in
  series with the circuit elements?
\item The Oscilloscope has two display axes.  What quantities are
  typically displayed on the two axes in sweep mode?
\item Describe how to measure the amplitude of a sine wave signal in
  sweep mode.  How does that differ from the peak-to-peak and RMS
  voltages? 
\item Describe how to measure the frequency of a sine wave signal in
  sweep mode.
\item What is meant by triggering?
\end{enumerate}

\newpage

\section*{Post-Lab Exercises}

\begin{enumerate}
\item Format and submit your data and calculation sheets from the lab
  manual.
\item Discuss briefly whether you have met the objectives of the lab
  exercises. 
\end{enumerate}

\end{document}

%%% Local Variables: 
%%% mode: latex
%%% TeX-master: t
%%% End: 
