\documentclass[12pt]{article}

\usepackage{amsmath}
\usepackage{amssymb}
\usepackage{calc}
\usepackage{units}
\usepackage{graphicx}
\usepackage{subfig}
\usepackage[margin=1in]{geometry}
\usepackage{listings}
\usepackage[numbers,sort&compress]{natbib}
\usepackage{bm}
\usepackage{paralist}
\usepackage[draft]{fixme}

\title{Measuring Electric Phenomena:\\the Ammeter and Voltmeter}
%\author{Kevin R. Lynch}
%Kevin R. Lynch, based on an earlier lab by Duli Jain
%\date{2012-01-24}
\date{}

\begin{document}

\maketitle

\section{Objectives}
\label{sec:objectives}

\begin{enumerate}
\item To understand the use and operation of the Ammeter and
  Voltmeter in a simple direct current circuit, and
\item To verify Ohm's Law for the resistor.
\end{enumerate}

\section{Introduction}
\label{sec:introduction}

The German physicist, Georg Ohm, was the first to explore the
relationship between the current \textit{through} an object compared
to the voltage applied \textit{across} that object.  He published his
results, \textit{Die galvanische Kette, mathematisch
  bearbeitet}\footnote{The galvanic circuit investigated
  mathematically}, in 1827.  His results were not immediately
accepted, because his methods were so revolutionary, and challenged
the accepted requirements of scientific reasoning of his day; science
is a human endeavor, and this result has a fascinating backstory that
I urge you to investigate.

To investigate the properties of voltage and current, and the
relationship between the two, we need tools and equipment to do it.
We measure voltage with the \textit{voltmeter}, and current with the
ampmeter or \textit{ammeter}.  In a later lab, you will study the
detailed properties of ideal meters, and contrast them with the real
meters that we can actually build.  In this lab, however, you will
learn the proper use of these devices while investigating Ohm's Law.

\section{Theory}
\label{sec:theory}

Ohm showed that a \textit{steady} current was caused by a
\textit{constant} voltage, and that were directly proportional to each
other, and scaled with the length of the \textit{resistive} element
through which the current flowed.  Today, we express this relationship
mathematically as $V/I \propto 1$, where $V$ is the voltage (measured
in the SI system in \textit{volts}, $V$) and $I$ is the current
(measured in \textit{amperes}, $A$).  We give the \textit{constant of
  proportionality} the name \textit{resistance}, and the symbol $R$
\begin{gather*}
  \frac{V}{I} = R\ ,
\end{gather*}
Resistance is measured in \textit{ohms}, with the symbol $\Omega$.

In this lab, the resistance $R$ will be constant for a given object.
Later, we will investigate the limits of this relationship: under what
conditions does it hold true, when does it fail, and how can we
understand these properties as the results of microscopic physics.

Let's now introduce the voltmeter and the ammeter.  A voltmeter is
designed to measure the voltage \textit{across} a portion of a
circuit, while an ammeter is designed to measure the current
\textit{passing through} a particular point in the circuit.  While
these devices have distinct properties and uses, the ammeter and
voltmeter are such a basic part of the scientific measurement toolkit,
that the are usually found integrated into a single device called the
\textit{multimeter}.  Beware, however!  Switching a multimeter from
ammeter to voltmeter mode usually requires the inputs to be rewired;
failing to do so can result in damaging the device, or at the very
least, a blown fuse.  

In order to talk in more detail about these devices, we have to
describe the connections between them in a particular
\textit{electrical circuit} - the circuit diagram.\fixme{BLAH BLAH
  BLAH BLAH}.

\section{Procedures}
\label{sec:procedures}

In this lab, we will investigate the relationship between current and
voltage in the simplest possible circuit, consisting of a direct
current voltage source and a single resistance
(Figure~\ref{fig:simple}).\fixme{Nofigure}
\begin{figure}
  \centering
  
  \caption{The simplest direct current circuit.}
  \label{fig:simple}
\end{figure}
\begin{figure}
  \centering
  
  \caption{The direct current power supply.}
  \label{fig:dcps}
\end{figure}
\fixme{Nofigure}
\begin{figure}
  \centering
  \subfloat[][Settings for voltmeter mode]{picture here}\qquad
  \subfloat[][Settings for ammeter mode]{picture here}  
  \caption{The multimeter.}
  \label{fig:multimeter}
\end{figure}
\begin{enumerate}
\item First, construct the circuit shown in Figure~\ref{fig:simple},
  turn the \texttt{Voltage} knob on the power supply
  (Figure~\ref{fig:dcps}) to its minimum setting (fully counter
  clockwise), and the\texttt{Current} limiter knob to its maximum
  setting (fully clockwise).  Plug in and turn on the power supply.
\item Next, configure your first multimeter as a voltmeter: connect
  your test leads to the \texttt{COMMON} and \texttt{V} inputs, and
  select the \texttt{DCV} (Direct Current Voltage) Function.  Plug in
  and turn on the multimeter.
\item We'll now get a feel for how the meter works.  First, connect
  the test leads to opposite ends of the resistor - that is,
  \textit{across}) the resistor.  To make a measurement, you must
  select the correct \texttt{RANGE}; the meter will measure values
  from \unit[0]{V} up to the value specified above the \texttt{RANGE}
  button: \unit[2]{V} for the 2 button, \unit[20]{V} for the 20
  button, etc.  For maximum precision, always choose the smallest
  \texttt{RANGE} value larger than the voltage you are measuring.
  When in doubt, start at the largest setting, and work you way down.
  What happens when you go to far?

  Now, raise the voltage on the power supply; If you have connected
  the circuit and meter correctly, you should see the voltage display
  on the power supply increase from zero, and the value on the
  multimeter should roughly match that on the supply.  If not, you did
  something wrong and should ``debug'' your connections to make sure
  they are correct.  What happens when you swap the leads connected to
  the resistor?  Why?

  Turn the voltage down to zero before proceeding.
\item 
\fixme{Nofigure}
  \begin{figure}
    \centering
    
    \caption{The direct current circuit with ammeter inserted.}
    \label{fig:inserted}
  \end{figure}
  Next, configure the second multimeter as an ammeter: connect your
  test leads to the \texttt{COMMON} and \texttt{mA} inputs, and select
  the \texttt{DCmA} (Direct Current milliAmps) Function.  Again, you
  must choose an appropriate \texttt{RANGE}.  You measure current
  \textit{through} a device, which means you must insert the meter
  \textit{into} the circuit: you must ``cut'' the circuit, and
  ``splice'' the meter into the ``hole''.  In this case, you should
  insert the meter between the output of the power supply and the
  resistor.  Make sure the voltmeter is still connected across the
  resistor.  At this point, you should have recreated the circuit in
  Figure~\ref{fig:inserted}.  Again, slowly raise the voltage output,
  and observe the changes in the voltage and current values as
  measured on the respective meters.
\item We are ready to repeat Ohm's measurements!  Vary the voltage in
  small steps (say, ten steps from \unit[0]{V} to \unit[10]{V}), and
  record both the voltage and the corresponding current.  Don't forget
  the units!  Repeat the measurement; are your results consistent?
  Select a different resistor, and repeat your measurements.  Plot
  your data, $I$ vs $V$: if Ohm was right, this should be linear.  Is
  it?  What is the slope?  How is this related to $R$?
\item Finally, we will use the multimeter in \texttt{Resistance} mode
  to confirm our results.  In this mode, the multimeter performs the
  same experiment we just did in large: it applies a known voltage,
  measures the demand current, and deduces the resistance from these
  two values.  Completely disconnect the circuit.  Connect test leads
  to the \texttt{COMMON} and \texttt{$\Omega$} inputs, select the
  \texttt{OHMS} Function, and an appropriate value for the
  \texttt{RANGE}.  Connect the leads to opposite sides of the
  resistor, and record the value.  Does the value you measured with
  the above procedure agree with the value the meter measures
  directly?
\end{enumerate}

Make sure you clean up your work space, and return every item to the
condition and location you originally found them in!

\newpage

\section{Pre-Lab Questions}
\label{sec:prelab}

\newpage

\section{Post-Lab Questions}
\label{sec:postlab}



\end{document}

%%% Local Variables: 
%%% mode: latex
%%% TeX-master: t
%%% End: 
