\documentclass[12pt]{article}

\usepackage{amsmath}
\usepackage{amssymb}
\usepackage{calc}
\usepackage{units}
\usepackage{graphicx}
\usepackage[margin=1in]{geometry}
\usepackage{listings}
\usepackage[numbers,sort&compress]{natbib}
\usepackage{bm}
\usepackage{paralist}

\title{Measuring Electric Phenomena:\\the Ammeter and Voltmeter}
%\author{Kevin R. Lynch}
%Kevin R. Lynch, based on an earlier lab by Duli Jain
%\date{2012-01-24}
\date{}

\begin{document}

\maketitle

\section{Objectives}
\label{sec:objectives}

\begin{enumerate}
\item To understand the use and operation of the Ammeter and
  Voltmeter in a simple direct current circuit, and
\item To verify Ohm's Law for the resistor.
\end{enumerate}

\section{Introduction}
\label{sec:introduction}

The German physicist, Georg Ohm, was the first to explore the
relationship between the current \textit{through} an object compared
to the voltage applied \textit{across} that object.  He published his
results, \textit{Die galvanische Kette, mathematisch
  bearbeitet}\footnote{The galvanic circuit investigated
  mathematically}, in 1827.  His results were not immediately
accepted, because his methods were so revolutionary, and challenged
the accepted requirements of scientific reasoning of his day; science
is a human endeavor, and this result has a fascinating backstory that
I urge you to investigate.

Ohm showed that a \textit{steady} current was caused by a
\textit{constant} voltage, and that were directly proportional to each
other, and scaled with the length of the \textit{resistive} element
through which the current flowed.  Today, we express this relationship
mathematically as $V/I \propto 1$, where $V$ is the voltage (measured
in the SI system in \textit{volts}, $V$) and $I$ is the current
(measured in \textit{amperes}, $A$).  We give the \textit{constant of
  proportionality} the name \textit{resistance}, and the symbol $R$
\begin{gather*}
  \frac{V}{I} = R\ ,
\end{gather*}
Resistance is measured in \textit{ohms}, with the symbol $\Omega$.

In this lab, the resistance $R$ will be constant for a given object.
Later, we will investigate the limits of this relationship: under what
conditions does it hold true, when does it fail, and how can we
understand these properties as the results of microscopic physics.

To investigate the properties of voltage, current, and resistance, we
need tools and equipment to do it.  We measure voltage with the
\textit{voltmeter}, and current with the ampmeter or
\textit{ammeter}.  In this lab, you will learn the proper use of these
devices while investigating Ohm's Law.

\section{Theory}
\label{sec:theory}

\section{Procedures}
\label{sec:procedures}

two multimeters, a power supply, and a decade resistance box

\section{Pre-Lab Questions}
\label{sec:prelab}

\section{Cover Sheet}
\label{sec:coversheet}

\section{Data Sheet}
\label{sec:datasheet}

\section{Post-Lab Questions}
\label{sec:postlab}



\end{document}

%%% Local Variables: 
%%% mode: latex
%%% TeX-master: t
%%% End: 
